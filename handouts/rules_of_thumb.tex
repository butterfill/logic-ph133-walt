%!TEX TS-program = xelatex
%!TEX encoding = UTF-8 Unicode

\documentclass[11pt]{extarticle}
% extarticle is like article but can handle 8pt, 9pt, 10pt, 11pt, 12pt, 14pt, 17pt, and 20pt text

\def \ititle {Joint Action \& the Emergence of Mindreading}
\def \isubtitle {Lecture 2: Minimal Theory of Mind}
\def \iauthor {Stephen A. Butterfill and Ian Apperly}
\def \iemail{s.butterfill@warwick.ac.uk}
\date{}

\def \paperorient{landscape} %landscape
\def \papersize{a4paper}

\input{$HOME/Documents/submissions/preamble_steve_handout2}

\usepackage{fitch}

%itemize bullet should be dash
\renewcommand{\labelitemi}{$-$}

\begin{document}

\begin{multicols}{3}

\setlength\footnotesep{1em}

\bibpunct{}{}{,}{s}{}{,}  %use superscript TICS style bib

\bibliographystyle{newapa} %apalike

%\maketitle
%\tableofcontents






\begin{center}
{\Large
Rules of Thumb for Logic 1}


s.butterfill@warwick.ac.uk
\end{center}


There are exceptions to these rules of thumb.
But they are often useful.

\section{Proofs}

\subsection{Starting}
First ask, ‘Which $Elim$ rule can apply to this premise?’ for each premise.
Apply any $Elim$ rules you can first (except $\forall Elim$---see below).

Then ask, ‘Which $Intro$ rule would get me to this conclusion?’

If you still can’t get to the conclusion, try using $\lnot Intro$.
(You can do use $\lnot Intro$ even if the conclusion isn’t a negated sentence.
For example, if the conclusion is $A \lor B$, create a subproof with $\lnot(A \lor B)$ as premise, derive a contradiction, use $\lnot Intro$ to get $\lnot \lnot(A \lor B)$ then use $\lnot Elim$.)


\subsection{$\forall Elim$}
Use $\forall Elim$ as late as possible in your proof.

Only apply $\forall Elim$ using names that already occur in your proof.

\subsection{$\bot$}
Don’t use $\bot Elim$: you need $\lnot Intro$.

When using $\lor Elim$, if you are struggling to get two subproofs with matching conclusions try using $\bot Elim$ or $\lor Intro$.


\subsection{What to do with $\lnot$}
Having sentences that start with negation ($\lnot$) as premises is awkward.
Learning some standard proofs will help you.


\


If you have $\lnot (A \to B)$, you can get $A$ like this:
\begin{equation*}
    \begin{fitch}                 
        \fh \lnot (A \to B)            \\
        \fa \fh  \lnot A                    \\
        \fa \fa \fh A                \\
        \fa \fa \fa \bot            & $\bot Intro$: 2,3        \\
        \fa \fa \fa B                & $\bot Elim$: 4                        \\
        \fa \fa A \to B            & $\to Intro$: 3--5            \\
        \fa \fa \bot                & $\bot Intro$: 1, 7                    \\
        \fa \lnot \lnot A                & $\lnot Intro$: 2--7            \\
        \fa  A                            & $\lnot Elim$: 8        \\
    \end{fitch}
\end{equation*}



\



If you have $\lnot (A \to B)$, you can get $\lnot B$ like this:
\begin{equation*}
    \begin{fitch}                 
        \fh \lnot (A \to B)            \\
        \fa \fh  B                    \\
        \fa \fa \fh A                \\
        \fa \fa \fa B                & $Reit$: 2                        \\
        \fa \fa A \to B            & $\to Intro$: 3--4            \\
        \fa \fa \bot                & $\bot Intro$: 1, 7                    \\
        \fa \lnot B                & $\lnot Intro$: 2--6            \\
    \end{fitch}
\end{equation*}


If you have $\lnot (A \lor B )$, you can get $\lnot A$ like this:
\begin{equation*}
    \begin{fitch}                 
        \fh \lnot (A \lor B )            \\
        \fa \fh  A                    \\
        \fa \fa  (A \lor B )        & $\lor Intro$: 2                    \\
        \fa \fa \bot                & $\bot Intro$: 1, 3                    \\
        \fa \lnot A                & $\lnot Intro$: 2--4                \\
    \end{fitch}
\end{equation*}

%If you have $\lnot (A \lor B )$, you can also get $\lnot B$ using a proof just like the one above.




You can use $\lnot \exists x Blue(x)$ almost as if it were  $\forall x \lnot Blue(x)$:  you can get $\lnot Blue(b)$ like this:
\begin{equation*}
    \begin{fitch}                 
        \fh \lnot \exists x Blue(x)            \\
        \fa \fh  Blue(b)                    \\
        \fa \fa \exists x Blue(x)  & $\exists Intro$: 2                \\
        \fa \fa \bot                & $\bot Intro$: 1, 3                        \\
        \fa \lnot Blue(b)                & $\lnot Intro$: 2--4            \\
    \end{fitch}
\end{equation*}





\section{Translation}

Use $\forall$ with $\to$, e.g.\ 
\begin{quote}
    $\forall x (Square(x) \to Broken(x))$ 
\end{quote}
means all squares are broken.

Use $\exists$ with $\land$, e.g.\
\begin{quote}
    $\exists x (Square(x) \land Broken(x))$ 
\end{quote}
means some square is broken.

English sentences with mixed quantifiers are ambiguous (e.g.\ ‘There is a store for everything.’).


\footnotesize 
\bibliography{$HOME/endnote/phd_biblio}

\end{multicols}

\end{document}